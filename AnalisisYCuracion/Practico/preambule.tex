%\documentclass[12pt,a4paper,spanish]{report}
\documentclass[12pt,a4paper,spanish]{article}

\usepackage[spanish]{babel}
\usepackage[utf8]{inputenc}
\usepackage[T1]{fontenc}
\usepackage{amsmath}
\usepackage{graphicx}
\usepackage[retainorgcmds]{IEEEtrantools}
\usepackage{fancybox}
%\usepackage{hyperref} % páginas web e hipervínculos en el documento
\usepackage[hidelinks]{hyperref} % no dibuja el recuadro del vínculo
\usepackage[usenames]{color}
\usepackage{xcolor}
\usepackage{enumerate}
\usepackage{amsfonts,amssymb}
\usepackage{xfrac} % para fracciones		    <=== sfrac
%\usepackage[makeroom]{cancel} %para cancelar
%\usepackage{paralist}
%\usepackage{multirow}
\usepackage{fancyhdr}
\usepackage{mathtools} %cuadros en ecuaciones
\usepackage{tcolorbox} % " de color
%\usepackege[table,xcdraw]{xcolor} %se utiliza con
%definecolor{nombre}{HTML}{69A84F} <- color hexadecimal
%\usepackage{datetime}
%
%\usepackage[style=apa]{biblatex}
%\DefineBibliographyStrings{spanish}{andothers={\textit{\!et al.,}}}
%\usepackage{csquotes}

%\usepackage[backend=biber,citestyle=authoryear,bibstyle=authortitle,sorting=nyt]{biblatex}
%\usepackage{natbib}% citas tipo Harvard (autor,año)
%\usepackage[round,comma,authoryear]{natbib}% citas tipo Harvard (autor,año)
%\citestyle{apa}
%\usepackage{apalike}
%\usepackage{biblatex}
%\usepackage[fixlanguage]{babelbib}
%\usepackage{babelbib}
%\usepackage{tasks}    % para listas horizontales    <=== > task(s)
%%\usepackage{exsheets} % para exámenes		      <===

\usepackage{caption}
\captionsetup{font=small,labelfont=bf} %modifica los captions
%\captionsetup[table]{font=small,labelfont=bf} %modifica los captions
%\captionsetup[table]{font=small,labelsep=period,labelfont=bf} %modifica los captions
%\captionsetup[figure]{font=small,labelfont=bf} %modifica los captions
%\captionsetup[longtable]{font=small,labelsep=period,labelfont=bf,width=16.5cm}
%%\usepackage[subfolder]{gnuplottex} % cleanup
\usepackage{keyval,latexsym,ifthen,moreverb} %requerido por caption
\usepackage{listings} % para insertar código
% para figura en primera página
% src: https://tex.stackexchange.com/questions/167719/how-to-use-background-image-in-latex
%\usepackage{tikz}
% src: https://ipfs-sec.stackexchange.cloudflare-ipfs.com/tex/A/question/46280.html
\usepackage{transparent}
%\usepackage{wallpaper}
%\usepackage{eso-pic}
%\newcommand\BackgroundPic{%
%  \put(0,0){%
%  \parbox[b][\paperheight]{\paperwidth}{%
%  \vfill
%  \centering
  %{\transparent{0.4}%
%  \includegraphics[width=\paperwidth,height=\paperheight,%
%  keepaspectratio]{img/tapa.eps}%
  %}
%  \vfill
%  }}}

%============
\newcommand*\rot{\rotatebox{60}}

% para el estilo y forma de la página
\addtolength{\textwidth}{2.5cm}
\pagestyle{fancy}
\lhead{\bf{G32}}
\chead{}
\rhead{\sl Dos Santos, Gramajo, Rubio \& Coca}
\lfoot{\sl AEyCD}
\cfoot{}
\rfoot{\thepage}
\renewcommand{\headrulewidth}{0.5pt}
\renewcommand{\footrulewidth}{0.5pt}
%\linespread{1.11} %separación entre líneas

\textheight 23cm 
%\textwidth 165mm
\topmargin -1cm 
\oddsidemargin 0in
\evensidemargin 0in
%\evensidemargin -1.5in

\def\fecha#1{\sf\large
%  \newcommand\myToday{
    #1 de
    \ifcase\month\or Enero\or Febrero\or Marzo\or Abril\or Mayo\or Junio\or
    Julio\or Agosto\or Septiembre\or Octubre\or Noviembre\or Diciembre\fi\space de
    \number\year
  }

\newcommand\myToday{ 
  \ifcase\month\or Enero\or Febrero\or Marzo\or Abril\or Mayo\or Junio\or
  Julio\or Agosto\or Septiembre\or Octubre\or Noviembre\or Diciembre\fi\space de
  %\space\number\day,  
  \number\year}

% cuadros
\newcommand{\cuadro}[1]{%
%{\begin{center} \ovalbox{\begin{minipage}[c]{.9\textwidth}
{\begin{center} \framebox{\begin{minipage}[c]{.9\textwidth}
 \begin{center} 
    { #1}
  \end{center} \end{minipage} }
 \end{center}}
 }

%
\newcounter{problema}
%
\renewcommand{\baselinestretch}{1.5}
\newcommand{\prob}[1]{\stepcounter{problema} 
\noindent{\bf Problema Nº \arabic{problema}:} }

\newcommand{\probp}[1]{\stepcounter{problema} 
\noindent{\bf Problema Nº \arabic{problema}: {\it(#1 pts.)}} }

% cambio de Listing a Sintáxis
\renewcommand{\lstlistingname}{Sintáxis}


% enumerate environment
% fuente: http://en.wikibooks.org/wiki/LaTeX/List_Structures
%\let\oldenumerate\enumerate
%\renewcommand{\enumerate}{
%  \oldenumerate
%  \setlength{\itemsep}{-5pt}
%  \setlength{\parskip}{0pt}
%  \setlength{\parsep}{0pt}
%}




% redifine item command !!! usar este
%\newlength{\wideitemsep}
%\setlength{\wideitemsep}{.1\itemsep}
%\addtolength{\wideitemsep}{-7pt}
%\let\olditem\item
%\renewcommand{\item}{\setlength{\itemsep}{\wideitemsep}\olditem}

\addto\captionsspanish{%
  \renewcommand{\tablename}%
  {Tabla}
}

% redifine item command
%\newlength{\wideitemsep}
%\setlength{\wideitemsep}{.1\itemsep}
%\addtolength{\wideitemsep}{-7pt}
%\let\olditem\item
%\renewcommand{\item}{\setlength{\itemsep}{\wideitemsep}\olditem}


\makeatletter % para usar con tcolorbox
\let\Asol\Aboxed
\let\@Asol\@Aboxed
\patchcmd{\Asol}{\@Aboxed}{\@Asol}{}{}%
\patchcmd{\@Asol}{\boxed{#1#2}}{\fcolorbox{red}{yellow}{$\displaystyle #1#2$}}{}{}%
\makeatother

